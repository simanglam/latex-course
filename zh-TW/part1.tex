\documentclass{beamer}

\input{preamble.tex}

\subtitle{第一部:基礎}
\begin{document}

%%%%%%%%%%%%%%%%%%%%%%%%%%%%%%%%%%%%%%%%%%%%%%%%%%%%%%%%%%%%%%%%%%%%%%%%%%%%%%%
%%%%%%%%%%%%%%%%%%%%%%%%%%%%%%%%%%%%%%%%%%%%%%%%%%%%%%%%%%%%%%%%%%%%%%%%%%%%%%%
%%%%%%%%%%%%%%%%%%%%%%%%%%%%%%%%%%%%%%%%%%%%%%%%%%%%%%%%%%%%%%%%%%%%%%%%%%%%%%%
\begin{frame}
\titlepage
\end{frame}

%%%%%%%%%%%%%%%%%%%%%%%%%%%%%%%%%%%%%%%%%%%%%%%%%%%%%%%%%%%%%%%%%%%%%%%%%%%%%%%
%%%%%%%%%%%%%%%%%%%%%%%%%%%%%%%%%%%%%%%%%%%%%%%%%%%%%%%%%%%%%%%%%%%%%%%%%%%%%%%
%%%%%%%%%%%%%%%%%%%%%%%%%%%%%%%%%%%%%%%%%%%%%%%%%%%%%%%%%%%%%%%%%%%%%%%%%%%%%%%
\begin{frame}{為什麼使用 \LaTeX{}?}
\begin{itemize}
\item 高品質的文件
\begin{itemize}
\item 特別是跟數學有關的
\end{itemize}
%
\item 被科學家為科學家創造
\begin{itemize}
\item 龐大且活躍的使用者社群
\end{itemize}
%
\item 強大、靈活 --- 你甚至可以自行擴展
\begin{itemize}
\item 針對論文、報告、試算表\ldots\ 的巨集 \ldots
\end{itemize}
\end{itemize}
\end{frame}

%%%%%%%%%%%%%%%%%%%%%%%%%%%%%%%%%%%%%%%%%%%%%%%%%%%%%%%%%%%%%%%%%%%%%%%%%%%%%%%
%%%%%%%%%%%%%%%%%%%%%%%%%%%%%%%%%%%%%%%%%%%%%%%%%%%%%%%%%%%%%%%%%%%%%%%%%%%%%%%
%%%%%%%%%%%%%%%%%%%%%%%%%%%%%%%%%%%%%%%%%%%%%%%%%%%%%%%%%%%%%%%%%%%%%%%%%%%%%%%
\begin{frame}[fragile]{\LaTeX\ 是如何工作的?}
\begin{itemize}
\item 你將文件用純文字(plain text)與描述文字結構和意義的命令(\cmd{commands})
\item \texttt{latex} 將你的文字與命令轉換成格式優美的文件
\end{itemize}
\vskip 2ex
\begin{center}
\begin{minted}[frame=single]{latex}
The rain in Spain falls \emph{mainly} on the plain.
\end{minted}
\vskip 2ex
\tikz\node[single arrow,fill=gray,font=\ttfamily\bfseries,%
  rotate=270,xshift=-1em]{latex};
\vskip 2ex
\fbox{The rain in Spain falls \emph{mainly} on the plain.}
\end{center}
\end{frame}

%%%%%%%%%%%%%%%%%%%%%%%%%%%%%%%%%%%%%%%%%%%%%%%%%%%%%%%%%%%%%%%%%%%%%%%%%%%%%%%
%%%%%%%%%%%%%%%%%%%%%%%%%%%%%%%%%%%%%%%%%%%%%%%%%%%%%%%%%%%%%%%%%%%%%%%%%%%%%%%
%%%%%%%%%%%%%%%%%%%%%%%%%%%%%%%%%%%%%%%%%%%%%%%%%%%%%%%%%%%%%%%%%%%%%%%%%%%%%%%
\begin{frame}[fragile]{更多範例}%\ldots
\begin{exampletwoup}
\begin{itemize}
\item 茶
\item 牛奶
\item 餅乾
\end{itemize}
\end{exampletwoup}
\vskip 2ex
\begin{exampletwoup}
\begin{figure}
\includegraphics{gerbil}
\end{figure}
\end{exampletwoup}
\vskip 2ex
\begin{exampletwoup}
\begin{equation}
\alpha + \beta + 1
\end{equation}
\end{exampletwoup}

\tiny{Image license: \href{https://pixabay.com/en/animal-apple-attractive-beautiful-1239390/}{CC0}}
\end{frame}

%%%%%%%%%%%%%%%%%%%%%%%%%%%%%%%%%%%%%%%%%%%%%%%%%%%%%%%%%%%%%%%%%%%%%%%%%%%%%%%
%%%%%%%%%%%%%%%%%%%%%%%%%%%%%%%%%%%%%%%%%%%%%%%%%%%%%%%%%%%%%%%%%%%%%%%%%%%%%%%
%%%%%%%%%%%%%%%%%%%%%%%%%%%%%%%%%%%%%%%%%%%%%%%%%%%%%%%%%%%%%%%%%%%%%%%%%%%%%%%
\begin{frame}[fragile]{態度調整}

\begin{itemize}
\item 使用命令去描述他是什麼,而不是他是看起來怎樣
\item 專注在你的文件內容裡
\item 讓\LaTeX{}完成他的工作
\end{itemize}
\end{frame}

%%%%%%%%%%%%%%%%%%%%%%%%%%%%%%%%%%%%%%%%%%%%%%%%%%%%%%%%%%%%%%%%%%%%%%%%%%%%%%%
%%%%%%%%%%%%%%%%%%%%%%%%%%%%%%%%%%%%%%%%%%%%%%%%%%%%%%%%%%%%%%%%%%%%%%%%%%%%%%%
%%%%%%%%%%%%%%%%%%%%%%%%%%%%%%%%%%%%%%%%%%%%%%%%%%%%%%%%%%%%%%%%%%%%%%%%%%%%%%%
\section{基礎}

%%%%%%%%%%%%%%%%%%%%%%%%%%%%%%%%%%%%%%%%%%%%%%%%%%%%%%%%%%%%%%%%%%%%%%%%%%%%%%%
%%%%%%%%%%%%%%%%%%%%%%%%%%%%%%%%%%%%%%%%%%%%%%%%%%%%%%%%%%%%%%%%%%%%%%%%%%%%%%%
%%%%%%%%%%%%%%%%%%%%%%%%%%%%%%%%%%%%%%%%%%%%%%%%%%%%%%%%%%%%%%%%%%%%%%%%%%%%%%%
\subsection{正式開始}
\begin{frame}[fragile]{\insertsubsection}
\begin{itemize}
\item 最簡單的\LaTeX{}文件:
\inputminted[frame=single]{latex}{basics.tex}
\item 命令以\emph{反斜線}開頭的\keystrokebftt{\bs}.
\item 所有文件都以\cmdbs{documentclass}為開頭
\item \emph{引數}被花括號包圍\keystrokebftt{\{} \keystrokebftt{\}}告訴 \LaTeX{}我們想要的文件類型\bftt{article}.
\item 百分符號\keystrokebftt{\%}開始註解\emph{comment} --- \LaTeX{}會忽略該行剩餘的部分
\end{itemize}
\end{frame}

%%%%%%%%%%%%%%%%%%%%%%%%%%%%%%%%%%%%%%%%%%%%%%%%%%%%%%%%%%%%%%%%%%%%%%%%%%%%%%%
%%%%%%%%%%%%%%%%%%%%%%%%%%%%%%%%%%%%%%%%%%%%%%%%%%%%%%%%%%%%%%%%%%%%%%%%%%%%%%%
%%%%%%%%%%%%%%%%%%%%%%%%%%%%%%%%%%%%%%%%%%%%%%%%%%%%%%%%%%%%%%%%%%%%%%%%%%%%%%%
\begin{frame}[fragile]{使用\wllogo\insertsubsection{}}
\begin{itemize}
\item Overleaf是線上的\LaTeX 編譯器
\item 他自動編譯你的 \LaTeX{}文件並產出結果
\vskip 2em
\begin{center}
\fbox{\href{\wlnewdoc{basics.tex}}{%
點擊這裡以在\wllogo{}中打開範例文件}}
\\[1ex]\scriptsize{}
為得到最好的使用體驗,推薦使用 \href{http://www.google.com/chrome}{Google Chrome} 或 \href{http://www.mozilla.org/en-US/firefox/new/}{FireFox}
\end{center}
\vskip 2ex
\item 課程結束後建議在Overleaf上試試看解決小試身手
\item \textbf{不!你應該在課程中就嘗試}
\end{itemize}
\end{frame}

%%%%%%%%%%%%%%%%%%%%%%%%%%%%%%%%%%%%%%%%%%%%%%%%%%%%%%%%%%%%%%%%%%%%%%%%%%%%%%%
%%%%%%%%%%%%%%%%%%%%%%%%%%%%%%%%%%%%%%%%%%%%%%%%%%%%%%%%%%%%%%%%%%%%%%%%%%%%%%%
%%%%%%%%%%%%%%%%%%%%%%%%%%%%%%%%%%%%%%%%%%%%%%%%%%%%%%%%%%%%%%%%%%%%%%%%%%%%%%%
\subsection{基礎排版}
\begin{frame}[fragile]{\insertsubsection{}:英文}
\small
\begin{itemize}
\item 將你的文字打在 \cmdbegin{document}與\cmdend{document}之間
\item 大多數的時間就和word無異,只需普通的打出文件內容就好%怪怪的,要修
\begin{exampletwouptiny}
Words are separated by one or more
spaces.

Paragraphs are separated by one
or more blank lines.
\end{exampletwouptiny}
\item 原始碼中的空白並不代表真實的輸出
\begin{exampletwouptiny}
The   rain       in Spain
falls mainly on the plain.
\end{exampletwouptiny}
\end{itemize}
\end{frame}

%%%%%%%%%%%%%%%%%%%%%%%%%%%%%%%%%%%%%%%%%%%%%%%%%%%%%%%%%%%%%%%%%%%%%%%%%%%%%%%
%%%%%%%%%%%%%%%%%%%%%%%%%%%%%%%%%%%%%%%%%%%%%%%%%%%%%%%%%%%%%%%%%%%%%%%%%%%%%%%
%%%%%%%%%%%%%%%%%%%%%%%%%%%%%%%%%%%%%%%%%%%%%%%%%%%%%%%%%%%%%%%%%%%%%%%%%%%%%%%
\begin{frame}[fragile]{\insertsubsection{}:注意事項}
\small
\begin{itemize}
\item 引號比較刁鑽\\
使用反引號 \keystroke{\`{}}來打出左引號,撇號 \keystroke{\'{}}打出右引號
\begin{exampletwouptiny}
Single quotes: `text'.

Double quotes: ``text''.
\end{exampletwouptiny}

\item 一些常見的符號在\LaTeX\ 有特殊的意義\\[1ex]
\begin{tabular}{cl}
\keystrokebftt{\%} & 百分比記號              \\
\keystrokebftt{\#} & 井字號 \\
\keystrokebftt{\&} & 與                 \\
\keystrokebftt{\$} & 錢幣符號               \\
\end{tabular}
\item 如果你直接打出這些符號那你就只會得到錯誤,為了打出這些這些符號,你必須在每個符號前使用反斜槓來使這些字符\emph{跳脫}他們原本的意義 
\begin{exampletwoup}
\$\%\&\#!
\end{exampletwoup}
\end{itemize}
\end{frame}

%%%%%%%%%%%%%%%%%%%%%%%%%%%%%%%%%%%%%%%%%%%%%%%%%%%%%%%%%%%%%%%%%%%%%%%%%%%%%%%
%%%%%%%%%%%%%%%%%%%%%%%%%%%%%%%%%%%%%%%%%%%%%%%%%%%%%%%%%%%%%%%%%%%%%%%%%%%%%%%
%%%%%%%%%%%%%%%%%%%%%%%%%%%%%%%%%%%%%%%%%%%%%%%%%%%%%%%%%%%%%%%%%%%%%%%%%%%%%%%
\begin{frame}[fragile]{處理錯誤}
\begin{itemize}
\item \LaTeX{} 在編譯時也可能會出錯,如果它出錯了它將會產出錯誤訊息並停止編譯檔案,你必須將修復這個錯誤,否則\LaTeX{} 將不會產生任何的輸出
\item 舉個例子,如果你將\cmdbs{emph}誤拼成\cmdbs{meph}, \LaTeX{}會停止編譯並跳出 ``undefined control sequence''的錯誤,因為``meph''並不是已知的命令
\end{itemize}
\begin{block}{面對錯誤的的建議}
\begin{enumerate}
\item 不要慌張!錯誤經常發生的
\item 盡快將他們修復完畢 --- 如果是你剛剛打的東西造成了錯誤,就從該處開始吧!
\item 如過這裡有多個錯誤,先從第一個錯誤開始 --- 造成的原因還有可能在更上面
\end{enumerate}
\end{block}
\end{frame}

%%%%%%%%%%%%%%%%%%%%%%%%%%%%%%%%%%%%%%%%%%%%%%%%%%%%%%%%%%%%%%%%%%%%%%%%%%%%%%%
%%%%%%%%%%%%%%%%%%%%%%%%%%%%%%%%%%%%%%%%%%%%%%%%%%%%%%%%%%%%%%%%%%%%%%%%%%%%%%%
%%%%%%%%%%%%%%%%%%%%%%%%%%%%%%%%%%%%%%%%%%%%%%%%%%%%%%%%%%%%%%%%%%%%%%%%%%%%%%%
\begin{frame}[fragile]{牛刀小試}

\begin{block}{嘗試利用\LaTeX\ 為下面這段文字排版:
\footnote{\url{http://en.wikipedia.org/wiki/Economy_of_the_United_States}}}
In March 2006, Congress raised that ceiling an additional \$0.79 trillion to
\$8.97 trillion, which is approximately 68\% of GDP. As of October 4, 2008, the
``Emergency Economic Stabilization Act of 2008'' raised the current debt ceiling
to \$11.3 trillion.
\end{block}
\vskip 2ex
\begin{center}
\fbox{\href{\wlnewdoc{basics-exercise-1.tex}}{%
點擊以在\wllogo{}中開啟}}
\end{center}

\begin{itemize}
\item 提示:注意那些有特殊意義的符號
\item 嘗試完之後
\fbox{\href{\wlnewdoc{basics-exercise-1-solution.tex}}{%
點擊這裡來看解答}}.
\end{itemize}
\end{frame}

%%%%%%%%%%%%%%%%%%%%%%%%%%%%%%%%%%%%%%%%%%%%%%%%%%%%%%%%%%%%%%%%%%%%%%%%%%%%%%%
%%%%%%%%%%%%%%%%%%%%%%%%%%%%%%%%%%%%%%%%%%%%%%%%%%%%%%%%%%%%%%%%%%%%%%%%%%%%%%%
%%%%%%%%%%%%%%%%%%%%%%%%%%%%%%%%%%%%%%%%%%%%%%%%%%%%%%%%%%%%%%%%%%%%%%%%%%%%%%%
\subsection{數學排版}
\begin{frame}[fragile]{\insertsubsection{}:錢幣符號}
\begin{itemize}
\item 為什麼錢幣符號\keystrokebftt{\$}特別?因為我們用它來標示數學公式\\[1ex]
\begin{exampletwouptiny}
%並不好:
Let a and b be distinct positive
integers, and let c = a - b + 1.

%好多了:
Let $a$ and $b$ be distinct positive
integers, and let $c = a - b + 1$.
\end{exampletwouptiny}
\item 永遠記得使用一對錢幣符號 --- 一個開始一個結束
\item \LaTeX{} 會自己處理間隙,他會忽略你的空間
\begin{exampletwouptiny}
Let $y=mx+b$ be \ldots

Let $y = m x + b$ be \ldots
\end{exampletwouptiny}
\end{itemize}
\end{frame}

%%%%%%%%%%%%%%%%%%%%%%%%%%%%%%%%%%%%%%%%%%%%%%%%%%%%%%%%%%%%%%%%%%%%%%%%%%%%%%%
%%%%%%%%%%%%%%%%%%%%%%%%%%%%%%%%%%%%%%%%%%%%%%%%%%%%%%%%%%%%%%%%%%%%%%%%%%%%%%%
%%%%%%%%%%%%%%%%%%%%%%%%%%%%%%%%%%%%%%%%%%%%%%%%%%%%%%%%%%%%%%%%%%%%%%%%%%%%%%%
\begin{frame}[fragile]{\insertsubsection{}:標記}
\begin{itemize}
\item 使用\keystrokebftt{\^}來上標文字,底線\keystrokebftt{\_} 來下標文字
\begin{exampletwouptiny}
$y = c_2 x^2 + c_1 x + c_0$
\end{exampletwouptiny}
\vskip 2ex

\item 使用花括號 \keystrokebftt{\{} \keystrokebftt{\}}來標示上標文字的群組
\begin{exampletwouptiny}
$F_n = F_n-1 + F_n-2$     % 糟糕!

$F_n = F_{n-1} + F_{n-2}$ % 正常!
\end{exampletwouptiny}
\vskip 2ex

\item 這裡有一些常見的希臘字母與數學表示法
\begin{exampletwouptiny}
$\mu = A e^{Q/RT}$

$\Omega = \sum_{k=1}^{n} \omega_k$
\end{exampletwouptiny}
\end{itemize}
\end{frame}

%%%%%%%%%%%%%%%%%%%%%%%%%%%%%%%%%%%%%%%%%%%%%%%%%%%%%%%%%%%%%%%%%%%%%%%%%%%%%%%
%%%%%%%%%%%%%%%%%%%%%%%%%%%%%%%%%%%%%%%%%%%%%%%%%%%%%%%%%%%%%%%%%%%%%%%%%%%%%%%
%%%%%%%%%%%%%%%%%%%%%%%%%%%%%%%%%%%%%%%%%%%%%%%%%%%%%%%%%%%%%%%%%%%%%%%%%%%%%%%
\begin{frame}[fragile]{\insertsubsection{}:展示方程式}
\begin{itemize}
\item 如果他是巨大且恐怖的,將他獨立\emph{展示}出來,使用
\cmdbegin{equation} and \cmdend{equation}.\\[2ex]
\begin{exampletwouptiny}
The roots of a quadratic equation
are given by
\begin{equation}
x = \frac{-b \pm \sqrt{b^2 - 4ac}}
         {2a}
\end{equation}
where $a$, $b$ and $c$ are \ldots
\end{exampletwouptiny}
\vskip 1em
{\scriptsize 注意:大部分在數學環境中的空間會被\LaTeX{}忽略掉,但\LaTeX{} 無法處理在 equation 中的空白行 --- 不要在數學環境中加入空白行}
\end{itemize}
\end{frame}

%%%%%%%%%%%%%%%%%%%%%%%%%%%%%%%%%%%%%%%%%%%%%%%%%%%%%%%%%%%%%%%%%%%%%%%%%%%%%%%
%%%%%%%%%%%%%%%%%%%%%%%%%%%%%%%%%%%%%%%%%%%%%%%%%%%%%%%%%%%%%%%%%%%%%%%%%%%%%%%
%%%%%%%%%%%%%%%%%%%%%%%%%%%%%%%%%%%%%%%%%%%%%%%%%%%%%%%%%%%%%%%%%%%%%%%%%%%%%%%
\begin{frame}[fragile]{小插曲:環境}
\begin{itemize}
\item \bftt{equation}是個\emph{環境} --- 背景
\item 指令可能會在不同的環境中產出不同的效果
\begin{exampletwouptiny}
We can write
$ \Omega = \sum_{k=1}^{n} \omega_k $
in text, or we can write
\begin{equation}
  \Omega = \sum_{k=1}^{n} \omega_k
\end{equation}
to display it.
\end{exampletwouptiny}
\vskip 2ex
\item 即使我們使用相同的命令,我們還是可以發現,$\Sigma$在\bftt{equation}環境中比較大,並且上下標的位置也有所改變
\vskip 1em
{\scriptsize 事實上\bftt{\$...\$}與\cmdbegin{math}\bftt{...}\cmdend{math}是等價的}
\end{itemize}
\end{frame}

%%%%%%%%%%%%%%%%%%%%%%%%%%%%%%%%%%%%%%%%%%%%%%%%%%%%%%%%%%%%%%%%%%%%%%%%%%%%%%%
%%%%%%%%%%%%%%%%%%%%%%%%%%%%%%%%%%%%%%%%%%%%%%%%%%%%%%%%%%%%%%%%%%%%%%%%%%%%%%%
%%%%%%%%%%%%%%%%%%%%%%%%%%%%%%%%%%%%%%%%%%%%%%%%%%%%%%%%%%%%%%%%%%%%%%%%%%%%%%%
\begin{frame}[fragile]{小插曲:環境}
\begin{itemize}
\item \cmdbs{begin}與\cmdbs{end}命令被用來創造許多不同的環境
\vskip 2ex

\item \bftt{itemize}與\bftt{enumerate} 環境創造列表
\begin{exampletwouptiny}
\begin{itemize} %以實心圓圈為標示
\item Biscuits
\item Tea
\end{itemize}

\begin{enumerate} %以數字為標示
\item Biscuits
\item Tea
\end{enumerate}
\end{exampletwouptiny}
\end{itemize}
\end{frame}

%%%%%%%%%%%%%%%%%%%%%%%%%%%%%%%%%%%%%%%%%%%%%%%%%%%%%%%%%%%%%%%%%%%%%%%%%%%%%%%
%%%%%%%%%%%%%%%%%%%%%%%%%%%%%%%%%%%%%%%%%%%%%%%%%%%%%%%%%%%%%%%%%%%%%%%%%%%%%%%
%%%%%%%%%%%%%%%%%%%%%%%%%%%%%%%%%%%%%%%%%%%%%%%%%%%%%%%%%%%%%%%%%%%%%%%%%%%%%%%
\begin{frame}[fragile]{小插曲:Package}

\begin{itemize}
\item 至今為止我們都是使用\LaTeX\ 自帶的命令與環境
\item \emph{Packages}提供額外的命令與環境,目前有超過一千種的可用Packages,更讚的是他們都是免費的

\item 在\emph{導言區}使用\cmdbs{usepackage}命令以載入所有我們想要的Packages

\item 例子:美國數學家協會的\bftt{amsmath}
\begin{minted}[fontsize=\small,frame=single]{latex}
\documentclass{article}
\usepackage{amsmath} %導言區
\begin{document}
% 現在你可以在這裡使用amsmath提供的命令與環境
\end{document}
\end{minted}
\end{itemize}
\end{frame}

%%%%%%%%%%%%%%%%%%%%%%%%%%%%%%%%%%%%%%%%%%%%%%%%%%%%%%%%%%%%%%%%%%%%%%%%%%%%%%%
%%%%%%%%%%%%%%%%%%%%%%%%%%%%%%%%%%%%%%%%%%%%%%%%%%%%%%%%%%%%%%%%%%%%%%%%%%%%%%%
%%%%%%%%%%%%%%%%%%%%%%%%%%%%%%%%%%%%%%%%%%%%%%%%%%%%%%%%%%%%%%%%%%%%%%%%%%%%%%%
\begin{frame}[fragile]{\insertsubsection{}:使用\bftt{amsmath}的例子}
\begin{itemize}
\item 使用\bftt{equation*} (``equation-star'')創造未標數字的方程式
\begin{exampletwouptiny}
\begin{equation*}
  \Omega = \sum_{k=1}^{n} \omega_k
\end{equation*}
\end{exampletwouptiny}
\item \LaTeX{} 會將相鄰的字母視為相乘的變數,有時候會達不到你想要的效果,\bftt{amsmath}定義了許多常用的數學運算子
\begin{exampletwouptiny}
\begin{equation*} % 糟糕!
 min_{x,y} (1-x)^2 + 100(y-x^2)^2
\end{equation*}
\begin{equation*} % 好多了!
\min_{x,y}{(1-x)^2 + 100(y-x^2)^2}
\end{equation*}
\end{exampletwouptiny}
\item 你也可以利用\cmdbs{operatorname}來將文字轉換成運算子
\begin{exampletwouptiny}
\begin{equation*}
\beta_i =
\frac{\operatorname{Cov}(R_i, R_m)}
     {\operatorname{Var}(R_m)}
\end{equation*}
\end{exampletwouptiny}
\end{itemize}
\end{frame}

%%%%%%%%%%%%%%%%%%%%%%%%%%%%%%%%%%%%%%%%%%%%%%%%%%%%%%%%%%%%%%%%%%%%%%%%%%%%%%%
%%%%%%%%%%%%%%%%%%%%%%%%%%%%%%%%%%%%%%%%%%%%%%%%%%%%%%%%%%%%%%%%%%%%%%%%%%%%%%%
%%%%%%%%%%%%%%%%%%%%%%%%%%%%%%%%%%%%%%%%%%%%%%%%%%%%%%%%%%%%%%%%%%%%%%%%%%%%%%%
\begin{frame}[fragile]{\insertsubsection{}:使用\bftt{amsmath}的例子}
\begin{itemize}{\small
\item 將一連串的方程式按照等號排列
\begin{align*}
(x+1)^3 &= (x+1)(x+1)(x+1) \\
        &= (x+1)(x^2 + 2x + 1) \\
        &= x^3 + 3x^2 + 3x + 1
\end{align*}
利用\bftt{align*}環境

% for whatever reason, this doesn't play well with the twoup environment
\begin{minted}[fontsize=\small,frame=single]{latex}
\begin{align*}
(x+1)^3 &= (x+1)(x+1)(x+1) \\
        &= (x+1)(x^2 + 2x + 1) \\
        &= x^3 + 3x^2 + 3x + 1
\end{align*}
\end{minted}
\item 與號\keystrokebftt{\&}將左欄(等號之前)與右欄(等號之後)分離
\item 雙重反斜槓\keystrokebftt{\bs}\keystrokebftt{\bs} 開始新行
}\end{itemize}
\end{frame}


%%%%%%%%%%%%%%%%%%%%%%%%%%%%%%%%%%%%%%%%%%%%%%%%%%%%%%%%%%%%%%%%%%%%%%%%%%%%%%%
%%%%%%%%%%%%%%%%%%%%%%%%%%%%%%%%%%%%%%%%%%%%%%%%%%%%%%%%%%%%%%%%%%%%%%%%%%%%%%%
%%%%%%%%%%%%%%%%%%%%%%%%%%%%%%%%%%%%%%%%%%%%%%%%%%%%%%%%%%%%%%%%%%%%%%%%%%%%%%%
\begin{frame}[fragile]{牛刀小試}

\begin{block}{嘗試利用\LaTeX\ 為下面這段文字排版:}
Let $X_1, X_2, \ldots, X_n$ be a sequence of independent and identically
distributed random variables with $\operatorname{E}[X_i] = \mu$ and
$\operatorname{Var}[X_i] = \sigma^2 < \infty$, and let
\begin{equation*}
S_n = \frac{1}{n}\sum_{i=1}^{n} X_i
\end{equation*}
denote their mean. Then as $n$ approaches infinity, the random variables
$\sqrt{n}(S_n - \mu)$ converge in distribution to a normal $N(0, \sigma^2)$.
\end{block}
\vskip 2ex
\begin{center}
\fbox{\href{\wlnewdoc{basics-exercise-2.tex}}{%
點擊這裡以在\wllogo{}中開啟}}
\end{center}
\begin{itemize}
\item 提示: $\infty$符號的命令是\cmdbs{infty}.
\item 如果你嘗試過了
\fbox{\href{\wlnewdoc{basics-exercise-2-solution.tex}}{%
點擊這裡來看解答}}.
\end{itemize}
\end{frame}
\subsection{中文排版}


%%%%%%%%%%%%%%%%%%%%%%%%%%%%%%%%%%%%%%%%%%%%%%%%%%%%%%%%%%%%%%%%%%%%%%%%%%%%%%%
%%%%%%%%%%%%%%%%%%%%%%%%%%%%%%%%%%%%%%%%%%%%%%%%%%%%%%%%%%%%%%%%%%%%%%%%%%%%%%%
%%%%%%%%%%%%%%%%%%%%%%%%%%%%%%%%%%%%%%%%%%%%%%%%%%%%%%%%%%%%%%%%%%%%%%%%%%%%%%%
\begin{frame}{\insertsubsection{}:基礎}
\begin{itemize}
\item 讓\LaTeX\ 看得懂unicode編碼
\begin{itemize}
\item 使用 xelatex 或 lualatex
\end{itemize}
\item 使用支持unicode的中文字體
\begin{itemize}
\item \href{https://www.overleaf.com/learn/latex/Questions/Which_OTF_or_TTF_fonts_are_supported_via_fontspec\%3F}{\wllogo\ 上的可用字體}
\item 在終端機使用fc-list -f ``\%{family}\textbackslash n'' :lang=zh > zhfont.txt以查看電腦上的可用字體
\end{itemize}
\item 使用支持中文的 Package
\begin{itemize}
\item 在 Xe\LaTeX\ 上用 xeCJK
\item 在 Lua\LaTeX\ 上用luatexja
\item 使用 CJK (過時了)
\end{itemize}
\end{itemize}
\end{frame}



%%%%%%%%%%%%%%%%%%%%%%%%%%%%%%%%%%%%%%%%%%%%%%%%%%%%%%%%%%%%%%%%%%%%%%%%%%%%%%%
%%%%%%%%%%%%%%%%%%%%%%%%%%%%%%%%%%%%%%%%%%%%%%%%%%%%%%%%%%%%%%%%%%%%%%%%%%%%%%%
%%%%%%%%%%%%%%%%%%%%%%%%%%%%%%%%%%%%%%%%%%%%%%%%%%%%%%%%%%%%%%%%%%%%%%%%%%%%%%%
\begin{frame}[fragile]{\insertsubsection{}:xeCJK}
\begin{itemize}
\item 最簡單的用法:\\
\begin{minted}[frame=single]{latex}
\documentclass{article}
\usepackage{xeCJK}%使用xeCJK package
\setCJKmainfont{TW-Kai}%設定主要字體
\begin{document}
你好!世界%現在就可以使用中文了
\end{document}
\end{minted}
\item 請記得使用Xe\LaTeX\ 編譯
\item 除了主要字體外,斜體與等寬字體需另行設定,\cmd{setCJKsansfont}設定斜體,\cmd{setCJKmonofont}設定等寬字體
\item 更多的技巧請參考\href{https://ftp.ntou.edu.tw/ctan/macros/xetex/latex/xecjk/xeCJK.pdf}{ xeCJK Package 使用手冊}
\end{itemize}
\end{frame}


%%%%%%%%%%%%%%%%%%%%%%%%%%%%%%%%%%%%%%%%%%%%%%%%%%%%%%%%%%%%%%%%%%%%%%%%%%%%%%%
%%%%%%%%%%%%%%%%%%%%%%%%%%%%%%%%%%%%%%%%%%%%%%%%%%%%%%%%%%%%%%%%%%%%%%%%%%%%%%%
%%%%%%%%%%%%%%%%%%%%%%%%%%%%%%%%%%%%%%%%%%%%%%%%%%%%%%%%%%%%%%%%%%%%%%%%%%%%%%%
\begin{frame}[fragile]{\insertsubsection{}:luatexja}
\begin{itemize}
\item 最簡單的用法:\\
\begin{minted}[frame=single]{latex}
\documentclass{article}
\usepackage{luatexja}%使用luatexja package
%\usepackage{luatexja-fontspec}
%加上-fontspec才可以設定主要字體
\setmainjfont{TW-Kai}%設定主要字體
\begin{document}
你好!世界%現在就可以使用中文了
\end{document}
\end{minted}
\item 請記得使用Lua\LaTeX\ 編譯
\item 此Package原先是專為日文設計的
\item 更多的技巧請參考\href{https://ftp.ntou.edu.tw/ctan/macros/luatex/generic/luatexja/doc/luatexja-en.pdf}{ luatexja Package 使用手冊}
\end{itemize}
\end{frame}
%%%%%%%%%%%%%%%%%%%%%%%%%%%%%%%%%%%%%%%%%%%%%%%%%%%%%%%%%%%%%%%%%%%%%%%%%%%%%%%
%%%%%%%%%%%%%%%%%%%%%%%%%%%%%%%%%%%%%%%%%%%%%%%%%%%%%%%%%%%%%%%%%%%%%%%%%%%%%%%
%%%%%%%%%%%%%%%%%%%%%%%%%%%%%%%%%%%%%%%%%%%%%%%%%%%%%%%%%%%%%%%%%%%%%%%%%%%%%%%
\begin{frame}{Part 1的總結}
\begin{itemize}
\item 恭喜!你已經學會了如何\ldots
\begin{itemize}
\item 利用\LaTeX\ 排版
\item 使用多種不同的指令
\item 處理發生的問題
\item 排版美麗的數學公式
\item 使用多種不同的環境
\item 使用 Package
\item 使用中文
\end{itemize}
\item 這真驚艷
\item 在Part 2我們將探討如何使用\LaTeX{}寫下有結構的文件,包含了小節、交互引用、圖片和參考書目,下集待續。
\end{itemize}
\end{frame}

\end{document}