\documentclass{beamer}

\input{preamble.tex}

\subtitle{第一部:基礎}
\begin{document}

%%%%%%%%%%%%%%%%%%%%%%%%%%%%%%%%%%%%%%%%%%%%%%%%%%%%%%%%%%%%%%%%%%%%%%%%%%%%%%%
%%%%%%%%%%%%%%%%%%%%%%%%%%%%%%%%%%%%%%%%%%%%%%%%%%%%%%%%%%%%%%%%%%%%%%%%%%%%%%%
%%%%%%%%%%%%%%%%%%%%%%%%%%%%%%%%%%%%%%%%%%%%%%%%%%%%%%%%%%%%%%%%%%%%%%%%%%%%%%%
\begin{frame}
\titlepage
\end{frame}

%%%%%%%%%%%%%%%%%%%%%%%%%%%%%%%%%%%%%%%%%%%%%%%%%%%%%%%%%%%%%%%%%%%%%%%%%%%%%%%
%%%%%%%%%%%%%%%%%%%%%%%%%%%%%%%%%%%%%%%%%%%%%%%%%%%%%%%%%%%%%%%%%%%%%%%%%%%%%%%
%%%%%%%%%%%%%%%%%%%%%%%%%%%%%%%%%%%%%%%%%%%%%%%%%%%%%%%%%%%%%%%%%%%%%%%%%%%%%%%
\begin{frame}{為什麼使用 \LaTeX{}?}
\begin{itemize}
\item 高品質的文件
\begin{itemize}
\item 特別是跟數學有關的
\end{itemize}
%
\item 被科學家為科學家創造
\begin{itemize}
\item 龐大且活躍的使用者社群
\end{itemize}
%
\item 強大、靈活 --- 你甚至可以自行擴展
\begin{itemize}
\item 針對論文、報告、試算表\ldots\ 的巨集 \ldots
\end{itemize}
\end{itemize}
\end{frame}

%%%%%%%%%%%%%%%%%%%%%%%%%%%%%%%%%%%%%%%%%%%%%%%%%%%%%%%%%%%%%%%%%%%%%%%%%%%%%%%
%%%%%%%%%%%%%%%%%%%%%%%%%%%%%%%%%%%%%%%%%%%%%%%%%%%%%%%%%%%%%%%%%%%%%%%%%%%%%%%
%%%%%%%%%%%%%%%%%%%%%%%%%%%%%%%%%%%%%%%%%%%%%%%%%%%%%%%%%%%%%%%%%%%%%%%%%%%%%%%
\begin{frame}[fragile]{\LaTeX\ 是如何工作的?}
\begin{itemize}
\item 你將文件用純文字(plain text)與描述文字結構和意義的命令(\cmd{commands})
\item \texttt{latex} 將你的文字與命令轉換成格式優美的文件
\end{itemize}
\vskip 2ex
\begin{center}
\begin{minted}[frame=single]{latex}
The rain in Spain falls \emph{mainly} on the plain.
\end{minted}
\vskip 2ex
\tikz\node[single arrow,fill=gray,font=\ttfamily\bfseries,%
  rotate=270,xshift=-1em]{latex};
\vskip 2ex
\fbox{The rain in Spain falls \emph{mainly} on the plain.}
\end{center}
\end{frame}

%%%%%%%%%%%%%%%%%%%%%%%%%%%%%%%%%%%%%%%%%%%%%%%%%%%%%%%%%%%%%%%%%%%%%%%%%%%%%%%
%%%%%%%%%%%%%%%%%%%%%%%%%%%%%%%%%%%%%%%%%%%%%%%%%%%%%%%%%%%%%%%%%%%%%%%%%%%%%%%
%%%%%%%%%%%%%%%%%%%%%%%%%%%%%%%%%%%%%%%%%%%%%%%%%%%%%%%%%%%%%%%%%%%%%%%%%%%%%%%
\begin{frame}[fragile]{更多範例}%\ldots
\begin{exampletwoup}
\begin{itemize}
\item 茶
\item 牛奶
\item 餅乾
\end{itemize}
\end{exampletwoup}
\vskip 2ex
\begin{exampletwoup}
\begin{figure}
\includegraphics{gerbil}
\end{figure}
\end{exampletwoup}
\vskip 2ex
\begin{exampletwoup}
\begin{equation}
\alpha + \beta + 1
\end{equation}
\end{exampletwoup}

\tiny{Image license: \href{https://pixabay.com/en/animal-apple-attractive-beautiful-1239390/}{CC0}}
\end{frame}

%%%%%%%%%%%%%%%%%%%%%%%%%%%%%%%%%%%%%%%%%%%%%%%%%%%%%%%%%%%%%%%%%%%%%%%%%%%%%%%
%%%%%%%%%%%%%%%%%%%%%%%%%%%%%%%%%%%%%%%%%%%%%%%%%%%%%%%%%%%%%%%%%%%%%%%%%%%%%%%
%%%%%%%%%%%%%%%%%%%%%%%%%%%%%%%%%%%%%%%%%%%%%%%%%%%%%%%%%%%%%%%%%%%%%%%%%%%%%%%
\begin{frame}[fragile]{Attitude adjustment}

\begin{itemize}
\item 使用命令去描述他是什麼,而不是他是看起來怎樣
\item 專注在你的文件內容裡
\item 讓\LaTeX{}完成他的工作
\end{itemize}
\end{frame}

%%%%%%%%%%%%%%%%%%%%%%%%%%%%%%%%%%%%%%%%%%%%%%%%%%%%%%%%%%%%%%%%%%%%%%%%%%%%%%%
%%%%%%%%%%%%%%%%%%%%%%%%%%%%%%%%%%%%%%%%%%%%%%%%%%%%%%%%%%%%%%%%%%%%%%%%%%%%%%%
%%%%%%%%%%%%%%%%%%%%%%%%%%%%%%%%%%%%%%%%%%%%%%%%%%%%%%%%%%%%%%%%%%%%%%%%%%%%%%%
\section{基礎}

%%%%%%%%%%%%%%%%%%%%%%%%%%%%%%%%%%%%%%%%%%%%%%%%%%%%%%%%%%%%%%%%%%%%%%%%%%%%%%%
%%%%%%%%%%%%%%%%%%%%%%%%%%%%%%%%%%%%%%%%%%%%%%%%%%%%%%%%%%%%%%%%%%%%%%%%%%%%%%%
%%%%%%%%%%%%%%%%%%%%%%%%%%%%%%%%%%%%%%%%%%%%%%%%%%%%%%%%%%%%%%%%%%%%%%%%%%%%%%%
\subsection{準備開始}
\begin{frame}[fragile]{\insertsubsection}
\begin{itemize}
\item 最簡單的\LaTeX{}文件:
\inputminted[frame=single]{latex}{basics.tex}
\item 指令以\emph{反斜線}開頭的\keystrokebftt{\bs}.
\item 所有文件都以\cmdbs{documentclass}為開頭
\item \emph{引數}被花括號包圍\keystrokebftt{\{} \keystrokebftt{\}}告訴 \LaTeX{}我們想要的文件類型\bftt{article}.
\item 百分符號\keystrokebftt{\%}開始註解\emph{comment} --- \LaTeX{}會忽略該行剩餘的部分
\end{itemize}
\end{frame}

%%%%%%%%%%%%%%%%%%%%%%%%%%%%%%%%%%%%%%%%%%%%%%%%%%%%%%%%%%%%%%%%%%%%%%%%%%%%%%%
%%%%%%%%%%%%%%%%%%%%%%%%%%%%%%%%%%%%%%%%%%%%%%%%%%%%%%%%%%%%%%%%%%%%%%%%%%%%%%%
%%%%%%%%%%%%%%%%%%%%%%%%%%%%%%%%%%%%%%%%%%%%%%%%%%%%%%%%%%%%%%%%%%%%%%%%%%%%%%%
\begin{frame}[fragile]{\insertsubsection{} 與 \wllogo}
\begin{itemize}
\item Overleaf是線上的\LaTeX 編譯器
\item 他自動編譯你的 \LaTeX{}文件並產出結果
\vskip 2em
\begin{center}
\fbox{\href{\wlnewdoc{basics.tex}}{%
點擊這裡以在\wllogo{}中打開範例文件}}
\\[1ex]\scriptsize{}
為得到最好的使用體驗,推薦使用 \href{http://www.google.com/chrome}{Google Chrome} 或 \href{http://www.mozilla.org/en-US/firefox/new/}{FireFox}.
\end{center}
\vskip 2ex
\item As we go through the following slides, try out the examples by typing them
into the example document on Overleaf.
\item \textbf{No really, you should try them out as we go!}
\end{itemize}
\end{frame}

%%%%%%%%%%%%%%%%%%%%%%%%%%%%%%%%%%%%%%%%%%%%%%%%%%%%%%%%%%%%%%%%%%%%%%%%%%%%%%%
%%%%%%%%%%%%%%%%%%%%%%%%%%%%%%%%%%%%%%%%%%%%%%%%%%%%%%%%%%%%%%%%%%%%%%%%%%%%%%%
%%%%%%%%%%%%%%%%%%%%%%%%%%%%%%%%%%%%%%%%%%%%%%%%%%%%%%%%%%%%%%%%%%%%%%%%%%%%%%%
\subsection{基礎排版}
\begin{frame}[fragile]{\insertsubsection{}}
\small
\begin{itemize}
\item 將你的文字打在 \cmdbegin{document}與\cmdend{document}之間
\item 在大多分的時間,你就正常的打字%怪怪的,要修
\begin{exampletwouptiny}
文字被一個 或更多  的空格分離

段落被一個或更多的空白行分離
\end{exampletwouptiny}
\item Space in the source file is collapsed in the output.
\begin{exampletwouptiny}
The   rain       in Spain
falls mainly on the plain.
\end{exampletwouptiny}
\end{itemize}
\end{frame}

%%%%%%%%%%%%%%%%%%%%%%%%%%%%%%%%%%%%%%%%%%%%%%%%%%%%%%%%%%%%%%%%%%%%%%%%%%%%%%%
%%%%%%%%%%%%%%%%%%%%%%%%%%%%%%%%%%%%%%%%%%%%%%%%%%%%%%%%%%%%%%%%%%%%%%%%%%%%%%%
%%%%%%%%%%%%%%%%%%%%%%%%%%%%%%%%%%%%%%%%%%%%%%%%%%%%%%%%%%%%%%%%%%%%%%%%%%%%%%%
\begin{frame}[fragile]{\insertsubsection{}:注意事項}
\small
\begin{itemize}
\item Quotation marks are a bit tricky:\\
use a backtick \keystroke{\`{}} on the left and an apostrophe \keystroke{\'{}} on the right.
\begin{exampletwouptiny}
Single quotes: `text'.

Double quotes: ``text''.
\end{exampletwouptiny}

\item Some common characters have special meanings in \LaTeX:\\[1ex]
\begin{tabular}{cl}
\keystrokebftt{\%} & percent sign              \\
\keystrokebftt{\#} & hash (pound / sharp) sign \\
\keystrokebftt{\&} & ampersand                 \\
\keystrokebftt{\$} & dollar sign               \\
\end{tabular}
\item If you just type these, you'll get an error. If you want one to appear in
the output, you have to \emph{escape} it by preceding it with a backslash.
\begin{exampletwoup}
\$\%\&\#!
\end{exampletwoup}
\end{itemize}
\end{frame}

%%%%%%%%%%%%%%%%%%%%%%%%%%%%%%%%%%%%%%%%%%%%%%%%%%%%%%%%%%%%%%%%%%%%%%%%%%%%%%%
%%%%%%%%%%%%%%%%%%%%%%%%%%%%%%%%%%%%%%%%%%%%%%%%%%%%%%%%%%%%%%%%%%%%%%%%%%%%%%%
%%%%%%%%%%%%%%%%%%%%%%%%%%%%%%%%%%%%%%%%%%%%%%%%%%%%%%%%%%%%%%%%%%%%%%%%%%%%%%%
\begin{frame}[fragile]{Handling Errors}
\begin{itemize}
\item \LaTeX{} can get confused when it is trying to compile your document. If
it does, it stops with an error, which you must fix before it will produce
any output.
\item For example, if you misspell \cmdbs{emph} as \cmdbs{meph}, \LaTeX{} will
stop with an ``undefined control sequence'' error, because ``meph'' is not
one of the commands it knows.
\end{itemize}
\begin{block}{Advice on Errors}
\begin{enumerate}
\item Don't panic! Errors happen.
\item Fix them as soon as they arise --- if what you just typed caused an error,
you can start your debugging there.
\item If there are multiple errors, start with the first one --- the cause may
even be above it.
\end{enumerate}
\end{block}
\end{frame}

%%%%%%%%%%%%%%%%%%%%%%%%%%%%%%%%%%%%%%%%%%%%%%%%%%%%%%%%%%%%%%%%%%%%%%%%%%%%%%%
%%%%%%%%%%%%%%%%%%%%%%%%%%%%%%%%%%%%%%%%%%%%%%%%%%%%%%%%%%%%%%%%%%%%%%%%%%%%%%%
%%%%%%%%%%%%%%%%%%%%%%%%%%%%%%%%%%%%%%%%%%%%%%%%%%%%%%%%%%%%%%%%%%%%%%%%%%%%%%%
\begin{frame}[fragile]{牛刀小試}

\begin{block}{Typeset this in \LaTeX:
\footnote{\url{http://en.wikipedia.org/wiki/Economy_of_the_United_States}}}
In March 2006, Congress raised that ceiling an additional \$0.79 trillion to
\$8.97 trillion, which is approximately 68\% of GDP. As of October 4, 2008, the
``Emergency Economic Stabilization Act of 2008'' raised the current debt ceiling
to \$11.3 trillion.
\end{block}
\vskip 2ex
\begin{center}
\fbox{\href{\wlnewdoc{basics-exercise-1.tex}}{%
點擊以在\wllogo{}中開啟}}
\end{center}

\begin{itemize}
\item 提示:注意那些有特殊意義的符號
\item 嘗試完之後
\fbox{\href{\wlnewdoc{basics-exercise-1-solution.tex}}{%
點擊這裡來看我的解答}}.
\end{itemize}
\end{frame}

%%%%%%%%%%%%%%%%%%%%%%%%%%%%%%%%%%%%%%%%%%%%%%%%%%%%%%%%%%%%%%%%%%%%%%%%%%%%%%%
%%%%%%%%%%%%%%%%%%%%%%%%%%%%%%%%%%%%%%%%%%%%%%%%%%%%%%%%%%%%%%%%%%%%%%%%%%%%%%%
%%%%%%%%%%%%%%%%%%%%%%%%%%%%%%%%%%%%%%%%%%%%%%%%%%%%%%%%%%%%%%%%%%%%%%%%%%%%%%%
\subsection{Typesetting Mathematics}
\begin{frame}[fragile]{\insertsubsection{}: Dollar Signs}
\begin{itemize}
\item Why are dollar signs \keystrokebftt{\$} special? We use them to mark mathematics in text.\\[1ex]
\begin{exampletwouptiny}
% not so good:
Let a and b be distinct positive
integers, and let c = a - b + 1.

% much better:
Let $a$ and $b$ be distinct positive
integers, and let $c = a - b + 1$.
\end{exampletwouptiny}
\item Always use dollar signs in pairs --- one to begin the mathematics, and one
to end it.
\item \LaTeX{} handles spacing automatically; it ignores your spaces.
\begin{exampletwouptiny}
Let $y=mx+b$ be \ldots

Let $y = m x + b$ be \ldots
\end{exampletwouptiny}
\end{itemize}
\end{frame}

%%%%%%%%%%%%%%%%%%%%%%%%%%%%%%%%%%%%%%%%%%%%%%%%%%%%%%%%%%%%%%%%%%%%%%%%%%%%%%%
%%%%%%%%%%%%%%%%%%%%%%%%%%%%%%%%%%%%%%%%%%%%%%%%%%%%%%%%%%%%%%%%%%%%%%%%%%%%%%%
%%%%%%%%%%%%%%%%%%%%%%%%%%%%%%%%%%%%%%%%%%%%%%%%%%%%%%%%%%%%%%%%%%%%%%%%%%%%%%%
\begin{frame}[fragile]{\insertsubsection{}: Notation}
\begin{itemize}
\item Use caret \keystrokebftt{\^} for superscripts and underscore \keystrokebftt{\_} for subscripts.
\begin{exampletwouptiny}
$y = c_2 x^2 + c_1 x + c_0$
\end{exampletwouptiny}
\vskip 2ex

\item Use curly braces \keystrokebftt{\{} \keystrokebftt{\}} to group
superscripts and subscripts.
\begin{exampletwouptiny}
$F_n = F_n-1 + F_n-2$     % oops!

$F_n = F_{n-1} + F_{n-2}$ % ok!
\end{exampletwouptiny}
\vskip 2ex

\item There are commands for Greek letters and common notation.
\begin{exampletwouptiny}
$\mu = A e^{Q/RT}$

$\Omega = \sum_{k=1}^{n} \omega_k$
\end{exampletwouptiny}
\end{itemize}
\end{frame}

%%%%%%%%%%%%%%%%%%%%%%%%%%%%%%%%%%%%%%%%%%%%%%%%%%%%%%%%%%%%%%%%%%%%%%%%%%%%%%%
%%%%%%%%%%%%%%%%%%%%%%%%%%%%%%%%%%%%%%%%%%%%%%%%%%%%%%%%%%%%%%%%%%%%%%%%%%%%%%%
%%%%%%%%%%%%%%%%%%%%%%%%%%%%%%%%%%%%%%%%%%%%%%%%%%%%%%%%%%%%%%%%%%%%%%%%%%%%%%%
\begin{frame}[fragile]{\insertsubsection{}: Displayed Equations}
\begin{itemize}
\item If it's big and scary, \emph{display} it on its own line using
\cmdbegin{equation} and \cmdend{equation}.\\[2ex]
\begin{exampletwouptiny}
The roots of a quadratic equation
are given by
\begin{equation}
x = \frac{-b \pm \sqrt{b^2 - 4ac}}
         {2a}
\end{equation}
where $a$, $b$ and $c$ are \ldots
\end{exampletwouptiny}
\vskip 1em
{\scriptsize Caution: \LaTeX{} mostly ignores your spaces in mathematics, but it
can't handle blank lines in equations --- don't put blank lines in your
mathematics.}
\end{itemize}
\end{frame}

%%%%%%%%%%%%%%%%%%%%%%%%%%%%%%%%%%%%%%%%%%%%%%%%%%%%%%%%%%%%%%%%%%%%%%%%%%%%%%%
%%%%%%%%%%%%%%%%%%%%%%%%%%%%%%%%%%%%%%%%%%%%%%%%%%%%%%%%%%%%%%%%%%%%%%%%%%%%%%%
%%%%%%%%%%%%%%%%%%%%%%%%%%%%%%%%%%%%%%%%%%%%%%%%%%%%%%%%%%%%%%%%%%%%%%%%%%%%%%%
\begin{frame}[fragile]{Interlude: Environments}
\begin{itemize}
\item \bftt{equation} is an \emph{environment} --- a context.
\item A command can produce different output in different contexts.
\begin{exampletwouptiny}
We can write
$ \Omega = \sum_{k=1}^{n} \omega_k $
in text, or we can write
\begin{equation}
  \Omega = \sum_{k=1}^{n} \omega_k
\end{equation}
to display it.
\end{exampletwouptiny}
\vskip 2ex
\item Note how the $\Sigma$ is bigger in the \bftt{equation} environment, and
how the subscripts and superscripts change position, even though we used the
same commands.
\vskip 1em
{\scriptsize In fact, we could have written \bftt{\$...\$} as
\cmdbegin{math}\bftt{...}\cmdend{math}.}
\end{itemize}
\end{frame}

%%%%%%%%%%%%%%%%%%%%%%%%%%%%%%%%%%%%%%%%%%%%%%%%%%%%%%%%%%%%%%%%%%%%%%%%%%%%%%%
%%%%%%%%%%%%%%%%%%%%%%%%%%%%%%%%%%%%%%%%%%%%%%%%%%%%%%%%%%%%%%%%%%%%%%%%%%%%%%%
%%%%%%%%%%%%%%%%%%%%%%%%%%%%%%%%%%%%%%%%%%%%%%%%%%%%%%%%%%%%%%%%%%%%%%%%%%%%%%%
\begin{frame}[fragile]{Interlude: Environments}
\begin{itemize}
\item The \cmdbs{begin} and \cmdbs{end} commands are used to create many
different environments.
\vskip 2ex

\item The \bftt{itemize} and \bftt{enumerate} environments generate lists.
\begin{exampletwouptiny}
\begin{itemize} % for bullet points
\item Biscuits
\item Tea
\end{itemize}

\begin{enumerate} % for numbers
\item Biscuits
\item Tea
\end{enumerate}
\end{exampletwouptiny}
\end{itemize}
\end{frame}

%%%%%%%%%%%%%%%%%%%%%%%%%%%%%%%%%%%%%%%%%%%%%%%%%%%%%%%%%%%%%%%%%%%%%%%%%%%%%%%
%%%%%%%%%%%%%%%%%%%%%%%%%%%%%%%%%%%%%%%%%%%%%%%%%%%%%%%%%%%%%%%%%%%%%%%%%%%%%%%
%%%%%%%%%%%%%%%%%%%%%%%%%%%%%%%%%%%%%%%%%%%%%%%%%%%%%%%%%%%%%%%%%%%%%%%%%%%%%%%
\begin{frame}[fragile]{Interlude: Packages}

\begin{itemize}
\item All of the commands and environments we've used so far are built into
\LaTeX.

\item \emph{Packages} are libraries of extra commands and environments. There
are thousands of freely available packages.

\item We have to load each of the packages we want to use with a
\cmdbs{usepackage} command in the \emph{preamble}.

\item Example: \bftt{amsmath} from the American Mathematical Society.
\begin{minted}[fontsize=\small,frame=single]{latex}
\documentclass{article}
\usepackage{amsmath} % preamble
\begin{document}
% now we can use commands from amsmath here...
\end{document}
\end{minted}
\end{itemize}
\end{frame}

%%%%%%%%%%%%%%%%%%%%%%%%%%%%%%%%%%%%%%%%%%%%%%%%%%%%%%%%%%%%%%%%%%%%%%%%%%%%%%%
%%%%%%%%%%%%%%%%%%%%%%%%%%%%%%%%%%%%%%%%%%%%%%%%%%%%%%%%%%%%%%%%%%%%%%%%%%%%%%%
%%%%%%%%%%%%%%%%%%%%%%%%%%%%%%%%%%%%%%%%%%%%%%%%%%%%%%%%%%%%%%%%%%%%%%%%%%%%%%%
\begin{frame}[fragile]{\insertsubsection{}: Examples with \bftt{amsmath}}
\begin{itemize}
\item Use \bftt{equation*} (``equation-star'') for unnumbered equations.
\begin{exampletwouptiny}
\begin{equation*}
  \Omega = \sum_{k=1}^{n} \omega_k
\end{equation*}
\end{exampletwouptiny}
\item \LaTeX{} treats adjacent letters as variables multiplied together, which
is not always what you want. \bftt{amsmath} defines commands for many common
mathematical operators.
\begin{exampletwouptiny}
\begin{equation*} % bad!
 min_{x,y} (1-x)^2 + 100(y-x^2)^2
\end{equation*}
\begin{equation*} % good!
\min_{x,y}{(1-x)^2 + 100(y-x^2)^2}
\end{equation*}
\end{exampletwouptiny}
\item You can use \cmdbs{operatorname} for others.
\begin{exampletwouptiny}
\begin{equation*}
\beta_i =
\frac{\operatorname{Cov}(R_i, R_m)}
     {\operatorname{Var}(R_m)}
\end{equation*}
\end{exampletwouptiny}
\end{itemize}
\end{frame}

%%%%%%%%%%%%%%%%%%%%%%%%%%%%%%%%%%%%%%%%%%%%%%%%%%%%%%%%%%%%%%%%%%%%%%%%%%%%%%%
%%%%%%%%%%%%%%%%%%%%%%%%%%%%%%%%%%%%%%%%%%%%%%%%%%%%%%%%%%%%%%%%%%%%%%%%%%%%%%%
%%%%%%%%%%%%%%%%%%%%%%%%%%%%%%%%%%%%%%%%%%%%%%%%%%%%%%%%%%%%%%%%%%%%%%%%%%%%%%%
\begin{frame}[fragile]{\insertsubsection{}: Examples with \bftt{amsmath}}
\begin{itemize}{\small
\item Align a sequence of equations at the equals sign
\begin{align*}
(x+1)^3 &= (x+1)(x+1)(x+1) \\
        &= (x+1)(x^2 + 2x + 1) \\
        &= x^3 + 3x^2 + 3x + 1
\end{align*}
with the \bftt{align*} environment.

% for whatever reason, this doesn't play well with the twoup environment
\begin{minted}[fontsize=\small,frame=single]{latex}
\begin{align*}
(x+1)^3 &= (x+1)(x+1)(x+1) \\
        &= (x+1)(x^2 + 2x + 1) \\
        &= x^3 + 3x^2 + 3x + 1
\end{align*}
\end{minted}
\item An ampersand \keystrokebftt{\&} separates the left column (before the
$=$) from the right column (after the $=$).
\item A double backslash \keystrokebftt{\bs}\keystrokebftt{\bs} starts a new
line.
}\end{itemize}
\end{frame}


%%%%%%%%%%%%%%%%%%%%%%%%%%%%%%%%%%%%%%%%%%%%%%%%%%%%%%%%%%%%%%%%%%%%%%%%%%%%%%%
%%%%%%%%%%%%%%%%%%%%%%%%%%%%%%%%%%%%%%%%%%%%%%%%%%%%%%%%%%%%%%%%%%%%%%%%%%%%%%%
%%%%%%%%%%%%%%%%%%%%%%%%%%%%%%%%%%%%%%%%%%%%%%%%%%%%%%%%%%%%%%%%%%%%%%%%%%%%%%%
\begin{frame}[fragile]{牛刀小試}

\begin{block}{Typeset this in \LaTeX:}
Let $X_1, X_2, \ldots, X_n$ be a sequence of independent and identically
distributed random variables with $\operatorname{E}[X_i] = \mu$ and
$\operatorname{Var}[X_i] = \sigma^2 < \infty$, and let
\begin{equation*}
S_n = \frac{1}{n}\sum_{i=1}^{n} X_i
\end{equation*}
denote their mean. Then as $n$ approaches infinity, the random variables
$\sqrt{n}(S_n - \mu)$ converge in distribution to a normal $N(0, \sigma^2)$.
\end{block}
\vskip 2ex
\begin{center}
\fbox{\href{\wlnewdoc{basics-exercise-2.tex}}{%
Click to open this exercise in \wllogo{}}}
\end{center}
\begin{itemize}
\item Hint: the command for $\infty$ is \cmdbs{infty}.
\item Once you've tried,
\fbox{\href{\wlnewdoc{basics-exercise-2-solution.tex}}{%
click here to see my solution}}.
\end{itemize}
\end{frame}

%%%%%%%%%%%%%%%%%%%%%%%%%%%%%%%%%%%%%%%%%%%%%%%%%%%%%%%%%%%%%%%%%%%%%%%%%%%%%%%
%%%%%%%%%%%%%%%%%%%%%%%%%%%%%%%%%%%%%%%%%%%%%%%%%%%%%%%%%%%%%%%%%%%%%%%%%%%%%%%
%%%%%%%%%%%%%%%%%%%%%%%%%%%%%%%%%%%%%%%%%%%%%%%%%%%%%%%%%%%%%%%%%%%%%%%%%%%%%%%
\begin{frame}{End of Part 1}
\begin{itemize}
\item 恭喜!你已經學會了如何\ldots
\begin{itemize}
\item 在\LaTeX\ 中排版文字
\item 使用多種不同的指令
\item 處理發生的問題
\item 排版美麗的數學公式
\item 使用多種不同的環境
\item 使用巨集
\end{itemize}
\item 這真神奇
\item 在Part 2我們將探討如何使用\LaTeX{}寫下有結構的文件,包含了小節、交互引用、圖片和參考書目,下集待續。
\end{itemize}
\end{frame}

\end{document}